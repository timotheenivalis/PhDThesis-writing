\begin{zusammenfassung}
\addcontentsline{toc}{chapter}{Zusammenfassung}
\textbf{Diese Doktorarbeit untersucht die stochastischen und selektiven Ursachen der Variation in Fitnesskomponenten und deren evolution{\"a}ren Konsequenzen in einer freilebenden Nagetierpopulation. Sie zeigt die gegenw{\"a}rtige, genetische Evolution von K{\"o}rpermasse und entkoppelt klassische Selektionssch{\"a}tzungen von adaptiver Evolution.}

Das Herzst{\"u}ck der Evolutionsbiologie liegt im Verst{\"a}ndnis der Vielfalt von Organismen. W{\"a}hrend {\"u}ber 150 Jahren haben Forscher die Ursachen von intraspezifischer Variation dokumentiert, wie sie zur Artbildung beitr{\"a}gt und zum Zusammenpassen von Organismen mit ihrer Umwelt. Zunehmende Bedenken wegen der schnellen anthropogenischen Ver{\"a}nderungen haben in letzter Zeit eine erneute Erforschung, wie sich freilebende Populationen an Umweltver{\"a}nderungen anpassen, vorangetrieben. Dieser neue Fokus offenbart die Schwierigkeiten im Messen von nat{\"u}rlicher Selektion, die Entflechtung von Evolution und plastischen Ver{\"a}nderungen und dem Vorhersagen von evolution{\"a}ren Entwicklungsverl{\"a}ufen. Unter anderem gibt es nur wenige robuste Beispiele von gegenw{\"a}rtiger Evolution in freilebenden Populationen, die, die M{\"o}glichkeit, dass Evolution Populationen bei schnellen Ver{\"a}nderungen der Umweltbedingungen rettet, fraglich erscheinen l{\"a}sst. In dieser Doktorarbeit erforsche ich die Ursachen nat{\"u}rlicher Selektion und Evolution in einer freilebenden Population von Schneem{\"a}usen (\textit{Chionomys nivalis}). Dank 10 Jahren intensivem Individuen-basiertem Monitoring und Genotypisierung, beinhaltet der Erkenntnisstand dieser Population Lebensweise, morphologische Daten und einen hochaufgel{\"o}sten Stammbaum. Deswegen ist diese Population unter den besten weltweit verf{\"u}gbaren um Selektion und Evolution in Aktion zu messen. 

Trotzdem ist die Population ziemlich klein und neuerliche Publikationen legen nahe, dass das evolution{\"a}re Potential in kleinen Populationen effektiv von Stochastik in Fitnesskomponenten aufgehoben wird. Ich beurteile diese Methoden, die in diesen Publikationen benutzt wurden und demonstriere, dass die Variation in Fitnesskomponenten nicht ausschliesslich stochastisch ist. Kleine Populationen, einschliesslich diese Schneem{\"a}use, zeigen evolution{\"a}res Potential. 

Mit Kollaboratoren vergleiche ich dann vier h{\"a}ufig benutzte methodologische Ans{\"a}tze um die Anteile von Evolution, Plastizit{\"a}t und Demographie an der ph{\"a}notypischen Ver{\"a}nderung zu entflechten. Wir identifizieren wichtige Unstimmigkeiten zwischen den Ans{\"a}tzen, die teilweise vom Gebrauch von unterschiedlichen Definitionen, aber auch vom Besitz von intrinsisch unterschiedlichen F{\"a}higkeiten stammen. Unter den in Betracht gezogenen Ans{\"a}tzen kann nur quantitative Genetik genetische Ver{\"a}nderungen messen. 

Durch die Anwendung von quantitativ-genetischen Methoden an der Schneemaus-Population demonstriere ich, dass sich K{\"o}rpermasse {\"u}ber die Studiendauer adaptiv entwickelt. Ich zeige auf, dass ph{\"a}notypische Sch{\"a}tzungen von Selektion nicht genetische Evolution hervorsagen: weder die durchschnittliche Selektion noch die temporale Variation h{\"a}ngen mit der Rate genetischer Evolution zusammen. Das legt dar, dass die dominante, rein ph{\"a}notypische Methode zur Messung von Selektion stattdessen die Messung von Variation im Ern{\"a}hrungszustand riskiert. Dennoch habe ich quantitative Genetik zur Identifikation von Selektion verwendet und Selektionssch{\"a}tzungen erhalten, die  mit der beobachteten genetischen Ver{\"a}nderung {\"u}bereinstimmen. 

Diese Doktorarbeit weist gegenw{\"a}rtige Evolution in einer freilebenden Population nach und zeigt dass evolution{\"a}re Reaktionen auf Umweltver{\"a}nderungen  von rein ph{\"a}notypischen Methoden weder zuverl{\"a}ssig eingesch{\"a}tzt noch verstanden werden k{\"o}nnen; ein expliziter, genetischer Ansatz ist notwendig. 
\end{zusammenfassung}
