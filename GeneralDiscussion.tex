\setchapterpreamble[ur][.6\textwidth]{%
\dictum[Miyamoto Musashi, \textit{A Book of Five Rings} (circa 1645)]{%
It is difficult to understand the universe if you only study one planet.}\vskip1em}

\chapter[\texorpdfstring{Chapter 6 \\ General discussion}{Chapter 6 General discussion}]{General discussion}
\chaptermark{General discussion}
\label{chap:discu}

\section{Overview}
In this thesis, I investigated the causes and consequences of variation in fitness in a wild population. I showed that the variation in proxies for individual fitness is not purely stochastic, but is underlain by variation in latent fitness (chapitre \ref{chap:dynhet}). Besides, the variation in latent fitness has an additive genetic component, showing the presence of natural selection and of adaptive evolution in the snow vole population (chapter \ref{chap:stasis} and \ref{chap:flusel}).
I explored ways to decompose the causes of phenotypic changes and identified the animal model from quantitative genetics as a convenient tool to estimate evolution (chapter \ref{chap:decpop}).
Using this tool in various ways, I showed that body mass was an important contributor of variation in fitness proxies (chapter \ref{chap:stasis}), but not in a consistent way over time (chapter \ref{chap:flusel}). Nevertheless, body mass was a consistent contributor to variation in genetic variation for fitness (chapters \ref{chap:stasis} and \ref{chap:flusel}), and therefore, body mass evolved over the study period.

Below, I will discuss further the insight brought by this thesis and the remaining challenges, in three domains: the causes of variation, the measurement of selection, and the response of wild populations to environmental change. 

\section{The causes of phenotypic variation}
%direct G effect, but also indirect G effects
In this thesis, we have decomposed the phenotypic variation of morphological and life-history traits into components related to additive genetic effects, maternal effects or permanent environments. This decomposition was necessary and sufficient to measure the rate of evolution, but the more biological insight could be obtained by pushing the variance decomposition further. Thus, an emerging question in evolutionary quantitative genetics is the role of indirect genetic effects \parencite{OUHUIHOU}. 
Maternal effects have been studied extensively \parencite{ODJOJ}, but only recently have researchers started to measure the genetic variance in maternal effects \parencite{ejfkj}. These genetic maternal effects could provide extra evolutionary potential in addition to that of direct genetic variation \parencite{Macfarlane2015}. 
Beyond maternal effects, genetic conflicts, that is, genetic trade-offs between traits expressed in different individuals \parencite{Trivers1974}, could be a major constrain on evolution and could explain cases of evolutionary stasis. Thus, \cite{Kolliker2015} demonstrated that a genetic trade-off between offspring number and offspring size constrains the evolution of size in earrings, and \cite{Rollinson2015} presented qualitative evidence suggesting that this constraint is widespread among animals. 
In the snow voles, the idea is very relevant  I tested the idea qualitatively only, 

but the system is not ideal to measure the trade-off 
because the sampling of juvenile is incomplete and selective disappearance is likely to blur the trade-off signal \parencite{Hadfield2011}
on-going evolution \parencite{Bonnet2016}
In theory, the question could be rigorously approached with quantitative genetics, but nor the exact model to fit nor the modelling tools are published yet \parencite{Hadfield2012, Rollinson2015}. An experimental approach remains the only option to quantitatively test for a size-number genetic trade-off \parencite{Kolliker2015}, and such an approach appears impossible in a wild population such as Churwalden's snow voles. 

Finally kin selection, matrilinean female lineages


%NGS

%fondamental maintenance of g var
Still, our study is a snapshot, too short to expect significant loss of genetic variation. What would happen if the observed selective pressures would carry on? 
Fluctuating selection might contribute to the maintenance of genetic variation
New mutations certainly contribute, but probably not sufficient in general (charlesworth).
Population structure. 

\section{Measuring selection in the wild}

% experiments: limits

% our approach: find the target first! what does covary with fitness. Price equation might be tautological, but can be useful.

% body mass is a very difficult trait because obviously condition dependance. 

\section{The response of wild populations to environmental change}

%response to selection

% demographic response and evo rescue.
Within-population variation in fitness is relevant to the demographic response to environmental changes, either through the direct effects of phenoytpic variation (including genetic variation) on demographic rates \parencite{Kendall2011, vindenes2015, Plard2016}, or indirectly, following genetic changes \parencite{Chevin2010a, Turcotte2011, Schiffers2013a}. 

\section{Conclusion}


\printbibliography[heading=subbibliography]

%\subsection{Ultimate causes of variation in fitness}
%
%Among the fundamental questions of evolutionary biology, the question of the origin of variation in fitness is in the background of every chapter of this thesis, without being dealt with extensively. Lek paradox. 
%We explored the stochastic vs. deterministic components of variation in reproductive success and survival, but what we call fitness is a propensity and is not stochastic. We also described some of the proximal causes of selection on morphology. Still, our study is a snapshot, too short to expect significant loss of genetic variation. What would happen if the observed selective pressures would carry on? 
%Fluctuating selection might contribute to the maintenance of genetic variation, but it does not seem to be a good general explanation for those traits that are mostly observed under the same selective pressure (chapter 4). New mutations certainly contribute, but probably not sufficient in general (charlesworth). 
%Population structure also contributes. That might be an important limitation of quantitative genetic studies. Selection and evolution observed at the level of one population, might not represent the general dynamics of the species. Cf. Mandarte, lack of evidence for inbreeding avoidance seems paradoxical in a population with such high indreeding depression, but the island receives immigrants from populations probably much bigger and interconnected where inbreeding levels are lower. 
%Trade-offs, in particular between life-stages are an appealing explanation surprisingly little explored. 
%Density dependance (cf. r/K selection in tits), frequency dependence, cf. environmental degradation more fundamental? 
%
%In the end, selection can be all described in ecological terms, but one might make the somehow artificial distinction between environmental variables, extrinsic to the species, such as climate, predation... and intrinsic selective processes, sexual selection and within-species competition. 
%The former can be more readily forecast, 
%
%\subsection{Ecological causes of selection}
%
%Can we generalize our results? Or do the idiosyncraties of biological systems require an in-depth study of each situation? 
%e.g. Marmots Ozgul Vs. Tafani are actually very similar, and the discrepancies are well explained by a single parameter, the snow depth. So the first study actually highlights the second. A posteriori, it is difficult to see what is unpredictable about it.
%Understanding how selection depends on the ecological variables will certainly improve predictive power, in particular by improving our ways to measure selection (which is a necessary big step forward, cf chapter 3).
%
%\subsection{Molecular genetic variation}
%Maybe a way to identify relevant ecological factors. e.g. Susan Jonhston sheep? 
%Also a link to population genetics and long-term evolution. 
%Indeed, quantitative genetics works irrespective of the genetic architecture on a few generations, but then the genetic architecture becomes crucial on the long run.
%
%Quantitative genetics will probably remain the method of choice in small natural populations, where population size limits the detectability of genetic loci to the one with most dramatic effects, and limits the possibility to explain a significant proportion of genetic variation. 
%Only in rare cases do we expect major loci for fitness and fitness-related traits. Hunting for those will likely lead to a few interesting discoveries of overdominance and other negative frequency-dependence mechanisms, but the polymorphism at these loci is intrinsically unstable (cf. X and Withlock). 
%Of course, no need to oppose top-down and bottom-up approaches and maybe it is best to make them work in combination, cf. scrub jays pedigree+SNPs or inbreeding depression + realized inbreeding and heterozygosity. 
%
%\subsection{Evolutionary demography}
%Beyond the fundamental understand of the living, the study of evolution in the wild aims at predicting the fate of natural populations in the face of environmental change. This is fundamentally a demographic question, and non-evolutionary forces probably dominate these responses. Still, we do know that significant genetic change can happen on the same time-scale as demographic processes, but demographic models show resistance to integrating the complexity of genetics. On the other hand, quantitative genetics and population genetics are traditionally more concerned with evolutionary questions that lack a demographic component. 
%New conceptual and methodological developments and a few empirical results. Maybe we need to look at host-parasite and disease litterature. 
%Evolutionary rescue, lots of theory, little testing. 