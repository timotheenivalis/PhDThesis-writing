\setchapterpreamble[ur][.6\textwidth]{%
\dictum[Miyamoto Musashi, \textit{A Book of Five Rings} (circa 1645)]{%
It is difficult to understand the universe if you only study one planet.}\vskip1em}

\chapter[\texorpdfstring{Chapter 6 \\ General discussion}{Chapter 6 General discussion}]{General discussion}
\chaptermark{General discussion}
\label{chap:discu}

\section{Overview}
In this thesis, I investigated the causes and consequences of variation in fitness in a wild population. I showed that the variation in proxies for individual fitness is not purely stochastic, but is underlain by variation in latent fitness (chapitre \ref{chap:dynhet}). Besides, the variation in latent fitness has an additive genetic component, showing the presence of natural selection and of adaptive evolution in the snow vole population (chapter \ref{chap:stasis} and \ref{chap:flusel}).
I explored ways to decompose the causes of phenotypic changes and identified the animal model from quantitative genetics as a convenient tool to estimate evolution (chapter \ref{chap:decpop}).
Using this tool in various ways, I showed that body mass was an important contributor of variation in fitness proxies (chapter \ref{chap:stasis}), but not in a consistent way over time (chapter \ref{chap:flusel}). Nevertheless, body mass was a consistent contributor to variation in genetic variation for fitness (chapters \ref{chap:stasis} and \ref{chap:flusel}), and therefore, body mass evolved over the study period.

Below, I will discuss further the insight brought by this thesis and the remaining challenges, in understanding the causes of phenotypic variation and the response of wild populations to environmental change. 

\section{The causes of phenotypic variation}
\subsection{Indirect genetic effects}
In this thesis, I have decomposed the phenotypic variation of morphological and life-history traits into components related to additive genetic effects, maternal effects or permanent environments. This decomposition was sufficient to measure the rate of evolution of the direct genetic effects (chapter \ref{chap:stasis}), that is, the direct actions of an individual's genes on its own body.
Nevertheless, an individual's genes have effects reaching out beyond its body, to the environment, including other individuals \parencite{Dawkins1982}, whether it is through interactions between individuals (indirect genetic effects, e.g. maternal effects, \cite{McAdam2014}), or through the pleiotropic action of genes expressed in kin at different life-stages (e.g. genetic conflicts).

Indirect genetic effects could be an important component shaping selection and evolution in the snow vole population. Indeed, in the snow voles, genes within an individual are likely to affect the phenotype of another individual during at least two types of situations. First, related females tend to form clusters of territories, and the presence of kin could suppress reproduction in subordinate females \parencite{Garcia-Navas2016}. Moreover, as in all placental mammals, maternal effects on offspring phenotypes are prevalent from pregnancy to weaning. 
Maternal effects have been studied extensively in natural populations \parencite{Wolf2009}, but estimations of the genetic component of maternal effects remain scarce \parencite{McAdam2014}. Nevertheless, genetic maternal effects could provide extra evolutionary potential in addition to that of direct genetic variation \parencite{Mcglothlin2014, McAdam2014, Mcfarlane2015}. In the snow vole population, preliminary analyses showed the presence of additive genetic maternal effects for body mass (results not shown). Genetic maternal effects for mass could therefore be subject to selection and evolve adaptively. In chapter \ref{chap:stasis}, maternal genetic effects are not explicitly modelled, and their evolution is assigned to phenotypic plasticity. A full account of body-mass evolution should measure this evolution in addition to that of direct additive genetic effects. 

Besides indirect genetic effects, the effect of others' genes matters for evolution in the case of genetic conflicts, that is, genetic trade-off between traits expressed in different individuals.
For four decades, genetic conflicts between parents and offspring have been thought to be a major constraint on the evolution of size \parencite[since][]{Trivers1974}, but the idea resisted attempts to test it, despite suggestive evidence from behavioural studies \parencite{Kolliker2015}. \cite{Kolliker2015} demonstrated that a genetic trade-off between offspring number and offspring size constrains the evolution of size in earrings. \cite{Rollinson2015b} presented qualitative evidence suggesting that this constraint is widespread among animals and could be a general explanation for the evolutionary stasis of size. 
In chapter \ref{chap:stasis} we briefly explored the possibility that a genetic conflict constrains the evolution of body mass, and found suggestive evidence that it is not the case. The snow vole study system is not an ideal to test this hypothesis, however. First, we do not capture all juveniles\textemdash some die or emigrate before their first year\textemdash and cannot measure litter size accurately. Because mass is under selection in juveniles,  selective disappearance is likely to blur the trade-off signal \parencite{Hadfield2011}. Second, the size-number genetic trade-off is best described as an explanation of evolutionary stasis of size or mass, but mass is evolving in the snow vole population, making it more difficult to formulate an expectation for the genetic covariance between mass and litter size. 
Finally, it is in theory possible to measure the genetic trade-off using quantitative genetics, but nor the exact model to fit nor the modelling tools are published yet \parencite{Hadfield2012, Rollinson2015b}. An experimental approach remains the only option to quantitatively test for a size-number genetic trade-off \parencite{Kolliker2015}, and such an approach appears impossible in a wild population such as Churwalden's snow voles. 

\subsection{Molecular basis of genetic variation}
During this PhD, on several occasions, Dr Erik Postma and myself, considered using high-throughput genome sequencing \parencite{VanDijk2014} to sequence the snow vole population retrospectively (tissue is kept in $-80\circ$ freezers for most of the individuals trapped in the last ten years).
As of yet, we did not obtain the funding necessary, and I ran out of time to carry out work in the laboratory and to develop a bio-informatic pipeline. 
As I discussed in chapter \ref{chap:intro}, molecular approaches to measuring selection and evolution are in general inferior to quantitative genetic approaches. Nevertheless, individual-based genomic data could bring complementary insights to my empirical chapters.

To start with, individual-based genomic data could marginally improve the estimation of quantitative genetic parameters \parencite{Berenos2014} by: (i) allowing the use of realized relatedness in animal models, instead of the relatedness expected from the pedigree; and (ii) providing some relatedness information about individuals with unknown parents (for which there is no information at all in a pedigree). 
More importantly, individual-based genomic data would allow the identification of some of the genetic loci underlying phenotypic variation and quantitative evolution.
This task which is in general challenging in small populations \parencite{Wellenreuther2016}, but the snow vole population presents three rare advantages that would ease it considerably. First, at least one trait, body mass, was evolving during the last decade. 
Second, in natural populations, it is difficult to show that evolution at a genetic locus is due to selection and not only due to drift, because there is in general no null-expectation for the effect of drift under complex demographics and mating patterns. A pedigree provides such a null expectation. Simulating the random dropping of alleles down a pedigree therefore provides a statistically robust test for the effect of selection on each genetic locus. This method was successfully used to show contemporary adaptive evolution at 67 genetic loci in a wild population of Florida scrub-jays (Nancy Chen, Evolution conference, 2016, Austin, USA).
Third, once adaptively evolving loci are identified, it is possible to correlate their allelic variation to success and failure in various life-stages, and thus to pinpoint when selection occurs in life, and what kind of molecular mechanism selection acts on.
Altogether, individual-based genomic data could therefore refine not only our molecular understanding of phenotypic variation, but also the ecological nature of selection. 

%\subsection{Origin and maintenance of variation in fitness}
%%fondamental maintenance of g var
%Still, our study is a snapshot, too short to expect significant loss of genetic variation. What would happen if the observed selective pressures would carry on? 
%Fluctuating selection might contribute to the maintenance of genetic variation
%New mutations certainly contribute, but probably not sufficient in general (charlesworth).
%Population structure. 

\section{Predicting responses to environmental change}
Anthropogenic environmental change has triggered research aiming at understanding and predicting the response of natural populations to environmental change \parencite{parmesan2006, Chevin2012, Smallegange2013, Charmantier2014climate}.
Massive challenges hinder this research agenda. Already, the retrospective study of phenotypic and demographic responses generally remains inconclusive \parencite{Merila2001,McCarty2001, Charmantier2014climate, Brookfield2016} and, at the moment, prospective prediction seems out of reach in most cases. 
During my PhD, I confronted at least three challenges that must be tackled to improve the predictive abilities of evolutionary ecology. Below, I discuss the problems with measuring selection, predicting the response to selection, and integrating evolutionary and demographic responses.

\subsection{Measuring selection in the wild}
For over 150 years, natural selection has been known to cause the match between organisms and their environment, and biologists have attempted to understand its causes and mechanisms. More recently, the study of selection assumed the more applied goal as researchers hope to predict the response of natural populations to the selective pressures imposed by environmental change \parencite{Chevin2010a, Charmantier2014climate}.
The principle of natural selection is very simple: in a given environment, individuals with a phenotype that favours survival and fertility contribute more to the next generation. Given the level of research attention on such a simple process, it can be surprising to see how slowly the understanding of natural selection has developed, and how difficult its study remains. 
For most of the 20th century, the main brake to progresses was the lack of an unified framework to quantify selection in natural populations \parencite{Wade2006}. Such a framework progressively emerged, starting with covariance-based methods \parencite{Robertson1966, Price1970} which efficiently measure the total effect of selection. The most influential breakdown was the popularization of regression-based methods \parencite{Lande1979,Lande1983} which measures the proportional effect of selection per unit of phenotypic variation, and allows to decompose selection into the direct and indirect effects of selection on multiple traits \parencite{Broodie1995}.
Since then, these methods have provided thousands of estimates of selection in natural populations \parencite{Kingsolver2001,Stinchcombe2008,Kingsolver2012}, showing several general patterns. For instance, directional selection is stronger and more common than suggested by early evolutionists, whereas stabilizing selection appears to be rare, while fertility selection is generally stronger than viability selection \parencite{Kingsolver2012}.
The abundance of estimates of selection should not be mistaken for a good understanding of natural selection, however. The estimation of selection through regression-methods faces at least three challenges that might severely hamper their significance and explain the general absence of response to selection \parencite{Merila2001, Brookfield2016}.

First, to obtain an unbiased measure of selection, fitness should be regressed on the trait of interest. Fitness is rarely observable directly, and fitness proxies must be used instead. Many estimates of selection are computed on fitness components, for instance fertility and survival \parencite{Kingsolver2012}. In this case, the estimation of selection can be biased in the presence of a trade-off between fitness components \parencite{Thompson2011, Kingsolver2012, Brookfield2016}. Fortunately, this bias appears to be minor in general, with the exception of body mass \parencite{Kingsolver2011}.
For the empirical part of this thesis (chapter \ref{chap:stasis} and \ref{chap:flusel}), I used fitness proxies that attempted to include all fitness components in order to avoid such biases. Thus, I used lifetime reproductive success when measuring selection within a generation, and annual reproductive success plus twice survival when measuring selection within a year. These fitness proxies are imperfect since we do not capture all juveniles and a trade-off between early juvenile survival and reproduction could bias the selection estimation. Still, the estimation of evolution using Price equation (that is, selection on the genotype) and that using the trend in BLUPs for breeding values (that is, not using any information about selection nor fitness) agree qualitatively (chapter \ref{chap:stasis}), suggesting that the generational proxy for fitness is adequate. 

Second, it is possible to estimate the total effect of selection on a trait with selection differentials, but it is much more difficult to disentangle the causal selective effect of a trait from the indirect selection due to other traits. In theory, it is possible to disentangle direct and indirect selection by including all the traits under selection in the analysis \parencite{Lande1983}. In natural populations, however, it is impossible to know \emph{a priori} what traits are under selection, and often it is impossible to measure all relevant traits \parencite{Brookfield2016, Hadfield2008}. Furthermore, as more traits are included in a selection analysis, the statistical power to detect significant selection on any one trait decreases \parencite{Mitchell-Olds1987}.
I did detect significant indirect selection on body mass, but genetic correlations between the traits considered were such that the prediction of evolution was not affected by the inclusion of indirect selection (chapter \ref{chap:stasis}). Only three traits were tested, however, and we cannot exclude that body mass is not under any direct selective pressure. The evolution of body mass could be driven by selection on an unmeasured trait.
Nevertheless, this problem is irrelevant to the measures of total selection and evolution, on which chapters \ref{chap:stasis} and \ref{chap:flusel} rely.

Third, covariance-based and regression-based methods to estimate phenotypic selection essentially measure the statistical association between traits and relative fitness. Selection must however be a causal association, be it direct or indirect. If the association is entirely mediated by an environmental covariance between traits and fitness, there is no selection and no possibility of genetic response to selection \parencite{Price1989, Rausher1992}. 
Body mass, the main trait analysed in this thesis, is likely to be very sensitive to this source of bias. Indeed, a favourable environment\textemdash for instance food rich and lacking parasites\textemdash is likely to lead to larger mass, high survival, and high fertility. Accordingly, phenotypic estimates of natural selection on mass and size are overwhelmingly positive \parencite{Blanckenhorn2000, Kingsolver2012} and an excess of environmental covariance does underlie the apparent selection on mass in the snow voles (chapter \ref{chap:stasis}).
A solution to the problem is the experimental manipulation of the trait of interest. This can break the link between phenotype and individual quality and reveals the causal action of phenotype on fitness components \parencite[e.g.][]{Tinbergen2004, Tschirren2006}. Still, experimental manipulation is no without its own limitations.
Thus, manipulations are work intensive, time consuming and must be thought carefully in order to manipulate the trait of interest without affecting any other trait. Moreover, manipulations cannot easily be applied to all traits. The approach has been widely used to study selection on brood size, but it is not clear to me how one could manipulate body mass in a controlled way (that is, without accidentally affecting other traits).
My approach to the challenge of environmental covariation has been to use quantitative genetics to identify the target of natural selection (chapter \ref{chap:stasis}). After having shown on-going adaptive evolution, I decomposed phenotypic selection into an additive genetic and an environmental component, for various fitness components. I found that only juvenile viability selection showed an additive genetic component, and according to the Robertson-Price identity, was the source of adaptive evolution. Understanding the mechanism of this selection and measuring its strength was then a matter of hypothesis testing.
This approach could be used on other systems provided the presence of adaptive evolution. Nonetheless, it requires sufficient phenotypic and relatedness data to fit bivariate animal models. In addition, in the snow vole a single fitness drove evolution, but multiple fitness components could be involved, thus complicating the analysis. Finally, identifying the right fitness component(s) does not guarantee that the phenotypic mechanism of selection can be identified. A good understanding of the biological system will be necessary to formulate a reasonable hypothesis for the cause of selection. The testability of this hypothesis will also depend on data availability and quality, and will be subject to the limits of hypothesis testing approaches: there is always a risk of false positive, equal to the significance level chosen for the test, and a correlation does not prove causation.


\subsection{Evolutionary response}
Once a measure of phenotypic selection is obtained, it is straightforward to formulate a prediction of genetic response based on breeder's equation and on a heritability estimate \parencite{Lush1937}. We have already seen (chapter \ref{chap:stasis} and \ref{chap:flusel}) that such a prediction is often unreliable in natural populations, however.  Estimates of selection might not correspond to causal selection, and unmeasured selection acting on genetically correlated traits might constrain evolution. I have shown that estimating the genetic component of selection, or the rate of evolution, can test whether selection has been measured appropriately to be predictive (chapter \ref{chap:stasis}). 

Most attempts to understand the evolutionary response to environmental change do not measure genetic parameters, however. Thus, the alarming lack of evidence for evolutionary responses to climate change probably originates primarily from a lack of tests for genetic change \cite{Charmantier2014,Gienapp2014,Merila2014,Crozier2014}.
Ignoring the genetic properties (e.g. the heritability) of the trait of interest \parencite[e.g.][]{Forcada2014, Coulson2014, traill2014demography} easily leads to underestimating, or incorrectly dismiss, the potential to respond to selection and the actual evolutionary response \parencite{Nietlisbach2015, Chevin2015a, Pigeon2016}.
Similarly, the evolutionary potential of small populations was dismissed by population matrix simulations that ignored genetic-based arguments (see chapter \ref{chap:dynhet}. Moreover, methods based on phenotypic covariances do not distinguish between the presence and the absence of heritable variation, and cannot be used alone to predict an evolutionary response (chapter \ref{chap:decpop}).

Therefore, all the chapters of this thesis illustrate that a genetic approach, be it based on quantitative genetics or population genetics, is necessary to measure evolution, and can more reliably identify the selective causes and the constraints shaping adaptation. 
Attempts to understand the evolutionary dynamics of natural populations based on phenotypic observations only \parencite[e.g.][]{Smallegange2013} are a gamble, that might work on special occasions, but is unlikely to be reliable in general. 

\subsection{Demographic response to environmental change}
This thesis is almost exclusively concerned with traits and their evolutionary dynamics. 
In the context of understanding the response of natural population to environmental change, such an investigation is legitimate. Whether a trait distribution changes through demographic, plastic, or genetic mechanisms has different consequences on the fate of the population \parencite{Chevin2010}.
Nevertheless, for most applications, and to the eyes of the society, it is unimportant whether animal and plant populations respond to climate change primarily through migration, through plastic changes, or through evolution. The primary motivation of this research is to ascertain whether populations will persist or go extinct, and how managers can affect the outcome.
This question is primarily a demographic one. The evolutionary approach that was mine during this PhD is not sufficient to ascertain the fate of the snow vole population, although it might be a useful first step.


% demographic response and evo rescue.
Within-population variation in fitness is relevant to the demographic response to environmental changes, either through the direct effects of phenoytpic variation (including genetic variation) on demographic rates \parencite{Kendall2011, vindenes2015, Plard2016}, or indirectly, following genetic changes \parencite{Chevin2010a, Turcotte2011, Schiffers2013a}. 

\section{Conclusion}


\printbibliography[heading=subbibliography]

%\subsection{Ultimate causes of variation in fitness}
%
%Among the fundamental questions of evolutionary biology, the question of the origin of variation in fitness is in the background of every chapter of this thesis, without being dealt with extensively. Lek paradox. 
%We explored the stochastic vs. deterministic components of variation in reproductive success and survival, but what we call fitness is a propensity and is not stochastic. We also described some of the proximal causes of selection on morphology. Still, our study is a snapshot, too short to expect significant loss of genetic variation. What would happen if the observed selective pressures would carry on? 
%Fluctuating selection might contribute to the maintenance of genetic variation, but it does not seem to be a good general explanation for those traits that are mostly observed under the same selective pressure (chapter 4). New mutations certainly contribute, but probably not sufficient in general (charlesworth). 
%Population structure also contributes. That might be an important limitation of quantitative genetic studies. Selection and evolution observed at the level of one population, might not represent the general dynamics of the species. Cf. Mandarte, lack of evidence for inbreeding avoidance seems paradoxical in a population with such high indreeding depression, but the island receives immigrants from populations probably much bigger and interconnected where inbreeding levels are lower. 
%Trade-offs, in particular between life-stages are an appealing explanation surprisingly little explored. 
%Density dependance (cf. r/K selection in tits), frequency dependence, cf. environmental degradation more fundamental? 
%
%In the end, selection can be all described in ecological terms, but one might make the somehow artificial distinction between environmental variables, extrinsic to the species, such as climate, predation... and intrinsic selective processes, sexual selection and within-species competition. 
%The former can be more readily forecast, 
%
%\subsection{Ecological causes of selection}
%
%Can we generalize our results? Or do the idiosyncraties of biological systems require an in-depth study of each situation? 
%e.g. Marmots Ozgul Vs. Tafani are actually very similar, and the discrepancies are well explained by a single parameter, the snow depth. So the first study actually highlights the second. A posteriori, it is difficult to see what is unpredictable about it.
%Understanding how selection depends on the ecological variables will certainly improve predictive power, in particular by improving our ways to measure selection (which is a necessary big step forward, cf chapter 3).
%
%\subsection{Molecular genetic variation}
%Maybe a way to identify relevant ecological factors. e.g. Susan Jonhston sheep? 
%Also a link to population genetics and long-term evolution. 
%Indeed, quantitative genetics works irrespective of the genetic architecture on a few generations, but then the genetic architecture becomes crucial on the long run.
%
%Quantitative genetics will probably remain the method of choice in small natural populations, where population size limits the detectability of genetic loci to the one with most dramatic effects, and limits the possibility to explain a significant proportion of genetic variation. 
%Only in rare cases do we expect major loci for fitness and fitness-related traits. Hunting for those will likely lead to a few interesting discoveries of overdominance and other negative frequency-dependence mechanisms, but the polymorphism at these loci is intrinsically unstable (cf. X and Withlock). 
%Of course, no need to oppose top-down and bottom-up approaches and maybe it is best to make them work in combination, cf. scrub jays pedigree+SNPs or inbreeding depression + realized inbreeding and heterozygosity. 
%
%\subsection{Evolutionary demography}
%Beyond the fundamental understand of the living, the study of evolution in the wild aims at predicting the fate of natural populations in the face of environmental change. This is fundamentally a demographic question, and non-evolutionary forces probably dominate these responses. Still, we do know that significant genetic change can happen on the same time-scale as demographic processes, but demographic models show resistance to integrating the complexity of genetics. On the other hand, quantitative genetics and population genetics are traditionally more concerned with evolutionary questions that lack a demographic component. 
%New conceptual and methodological developments and a few empirical results. Maybe we need to look at host-parasite and disease litterature. 
%Evolutionary rescue, lots of theory, little testing. 