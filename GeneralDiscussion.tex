\setchapterpreamble[ur][.6\textwidth]{%
\dictum[Miyamoto Musashi, \textit{A Book of Five Rings} (circa 1645)]{%
It is difficult to understand the universe if you only study one planet.}\vskip1em}

\chapter[General discussion]{General discussion}

\section{Remaining challenges and future directions}

\subsection{Ultimate causes of variation in fitness}

Among the fundamental questions of evolutionary biology, the question of the origin of variation in fitness is in the background of every chapter of this thesis, without being dealt with extensively. Lek paradox. 
We explored the stochastic vs. deterministic components of variation in reproductive success and survival, but what we call fitness is a propensity and is not stochastic. We also described some of the proximal causes of selection on morphology. Still, our study is a snapshot, too short to expect significant loss of genetic variation. What would happen if the observed selective pressures would carry on? 
Fluctuating selection might contribute to the maintenance of genetic variation, but it does not seem to be a good general explanation for those traits that are mostly observed under the same selective pressure (chapter 4). New mutations certainly contribute, but probably not sufficient in general (charlesworth). 
Population structure also contributes. That might be an important limitation of quantitative genetic studies. Selection and evolution observed at the level of one population, might not represent the general dynamics of the species. Cf. Mandarte, lack of evidence for inbreeding avoidance seems paradoxical in a population with such high indreeding depression, but the island receives immigrants from populations probably much bigger and interconnected where inbreeding levels are lower. 
Trade-offs, in particular between life-stages are an appealing explanation surprisingly little explored. 
Density dependance (cf. r/K selection in tits), frequency dependence, cf. environmental degradation more fundamental? 

In the end, selection can be all described in ecological terms, but one might make the somehow artificial distinction between environmental variables, extrinsic to the species, such as climate, predation... and intrinsic selective processes, sexual selection and within-species competition. 
The former can be more readily forecast, 

\subsection{Ecological causes of selection}

Can we generalize our results? Or do the idiosyncraties of biological systems require an in-depth study of each situation? 
e.g. Marmots Ozgul Vs. Tafani are actually very similar, and the discrepancies are well explained by a single parameter, the snow depth. So the first study actually highlights the second. A posteriori, it is difficult to see what is unpredictable about it.
Understanding how selection depends on the ecological variables will certainly improve predictive power, in particular by improving our ways to measure selection (which is a necessary big step forward, cf chapter 3).

\subsection{Molecular genetic variation}
Maybe a way to identify relevant ecological factors. e.g. Susan Jonhston sheep? 
Also a link to population genetics and long-term evolution. 
Indeed, quantitative genetics works irrespective of the genetic architecture on a few generations, but then the genetic architecture becomes crucial on the long run.

Quantitative genetics will probably remain the method of choice in small natural populations, where population size limits the detectability of genetic loci to the one with most dramatic effects, and limits the possibility to explain a significant proportion of genetic variation. 
Only in rare cases do we expect major loci for fitness and fitness-related traits. Hunting for those will likely lead to a few interesting discoveries of overdominance and other negative frequency-dependence mechanisms, but the polymorphism at these loci is intrinsically unstable (cf. X and Withlock). 
Of course, no need to oppose top-down and bottom-up approaches and maybe it is best to make them work in combination, cf. scrub jays pedigree+SNPs or inbreeding depression + realized inbreeding and heterozygosity. 

\subsection{Evolutionary demography}
Beyond the fundamental understand of the living, the study of evolution in the wild aims at predicting the fate of natural populations in the face of environmental change. This is fundamentally a demographic question, and non-evolutionary forces probably dominate these responses. Still, we do know that significant genetic change can happen on the same time-scale as demographic processes, but demographic models show resistance to integrating the complexity of genetics. On the other hand, quantitative genetics and population genetics are traditionally more concerned with evolutionary questions that lack a demographic component. 
New conceptual and methodological developments and a few empirical results. Maybe we need to look at host-parasite and disease litterature. 
Evolutionary rescue, lots of theory, little testing. 