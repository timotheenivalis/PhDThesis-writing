\section{Missing data}
\label{app:miss}

When working with field data, completeness of the data is unlikely. Because the IPM, AM and GM are all based on fits, missing values for e.g. body size do not obstruct their application, but it will make their fits less reliable. Nevertheless, caution should be taken, especially when missing data are not randomly distributed. For instance, if some newborns die prior to measurement, it is possible that part of the viability selection is unobserved. As a consequence, estimates of selection or evolutionary change are biased downward. This problem is especially acute in the case of developing traits such as body mass or size \citep{Hadfield2008}. To some degree, the AM can deal with this so-called "missing fraction", because part of the missing information can be recovered from relatives \citep{Hadfield2008}. The APE will no longer provide an exact decomposition in the presence of missing data. However, an additional term, complementary to the missing part, can be added, as was done by \citet{Ozgul2009}.

Application of both the AM and the GM (which requires input from the AM) require information on relatedness among individuals in the form of a pedigree. However, also in the APE and IPM frameworks, the terms using the number of offspring ($r$) and the difference between parent and offspring ($d$) can not be estimated without information about relatedness.