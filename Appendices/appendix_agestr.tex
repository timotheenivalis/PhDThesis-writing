\newcommand{\avav}[3]{\overline{#1}(#3)\;\overline{#2}(#3)}
\newcommand{\cov}[2]{\text{cov}(#1,#2)}
\newcommand{\ages}{\sum_{a=1}^{\omega}}
\section{On demography in the EPE}
\label{app:ape2}
\subsection{Derivation of the APE}
The average trait value at time (t+1) is the sum of the trait values of all the individuals in the population at time t+1, divided by the number of individuals at that time ($N(t+1)$). That is, the trait values of all surviving individuals at time ($t+1$) plus the trait values of all newborn individuals. The former is given by the trait value at time t ($z_i(t)$) plus the growth of the individual at time t ($g_i(t)$). By multiplying each individuals expected trait value at time t+1 by a variable $s_i(t)$ that is $0$ for nonsurviving and $1$ for surviving individuals, we add up only the expected trait values for surviving individuals. Next we add to this term the trait values of the newborn offspring. The offspring of individual $i$ will have an average trait value that equals the parents trait value ($z_i(t)$) plus the average difference between a parents trait value and the trait value of its offspring ($d_i(t)$). Next this term has to be multiplied by the number of offspring the specific parent has ($r_i(t)$). Taking this all together we find:
\begin{equation}
\overline{Z}(t+1) = \frac{\sum_i^N s_i(t) \left( z_i(t) + g_i(t) \right) + \sum_i^N r_i(t) \left( z_i(t) + d_i(t) \right)}{N(t+1)}
\end{equation}
And thus, using the general expression $\sum_i^N b_i = N\overline{B}$:
\begin{equation}
\overline{Z}(t+1) = \frac{N(t)}{N(t+1)} \left( \overline{ZS}(t) + \overline{SG}(t) + \overline{RD}(t) + \overline{ZR}(t) \right)
\label{appape1}
\end{equation}
The next step in deriving the age structured price equation is introducing the age structure. If we define the fraction of individuals with age $a$ in the population ($c(a,t)$) as follows:
\begin{equation}
c(a,t) = \frac{N(a,t)}{N(t)}
\end{equation}
with $N(a,t)$ the number of individuals with age $a$ at time $t$ and $N(t)$ the total number of individuals. Given this quantity we can write for a general individual level defined variable $b_i$:
\begin{equation}
\overline{B}(t) = \sum_{a=1}^{\omega} c(a,t) \overline{B}(a,t)
\end{equation}
Here $\overline{B}(t)$ refers to the population wide average value of $b_i(t)$, whereas $\overline{B}(a,t)$ refers to the average value of $b_i(t)$ only of the individuals of age $a$. Besides the quantity $\omega$ is introduced. $\omega$ is the maximum age that individuals in the population can reach. Using this definition we can now rewrite equation \ref{appape1}:
\begin{equation}
\overline{Z}(t+1) = \sum_{a=1}^{\omega} \frac{N(t)c(a,t)}{N(t+1)} \left( \overline{ZS}(a,t) + \overline{SG}(a,t) + \overline{RD}(a,t) + \overline{ZR}(a,t) \right)
\label{appape2}
\end{equation}
Now we note that for two variables $B$ and $E$ in general:
\begin{equation}
\cov{B}{E}(a,t) = \overline{BE}(a,t) - \avav{B}{E}{a,t}
\end{equation}
and thus:
\begin{equation}
\overline{BE}(a,t) = \cov{B}{E}(a,t) + \avav{B}{E}{a,t}
\end{equation}
Which allows us to write equation \ref{appape2} as follows:
\begin{align}
\overline{Z}(t+1) & = \sum_{a=1}^{\omega} \frac{N(t)c(a,t)}{N(t+1)} ( \avav{Z}{S}{a,t} + \cov{S}{Z}(a,t) \\ & + \overline{SG}(a,t) + \avav{R}{D}{a,t} + \cov{R}{D}(a,t) + \avav{Z}{R}{a,t} + \cov{Z}{R}(a,t) ) \notag
\label{appape3}
\end{align}
We define the average population growth rate over all age classes $\overline{W}=\frac{N(t+1)}{N(t)}$. Besides we shift our focus to $\Delta \overline{Z}(t) = \overline{Z}(t+1) - \overline{Z}(t)$ and re-order the terms to arrive at:
\begin{align}
\Delta \overline{Z} = \overline{Z}(t+1) - \overline{Z}(t) & = \ages \frac{c(a,t)}{\overline{W}(t)} \avav{Z}{S}{a,t} - \overline{Z}(t) \\ & + \ages \frac{c(a,t)}{\overline{W}(t)} \left( \cov{S}{Z}(a,t)  + \overline{SG}(a,t)\right) \notag \\ & + \ages \frac{c(a,t)}{\overline{W}(t)} \left( \avav{R}{D}{a,t} + \cov{R}{D}(a,t) + \avav{Z}{R}{a,t} + \cov{Z}{R}(a,t) \right) \notag
\label{appape4}
\end{align}
In order to see that this is the age-structured price equation from \citet{Coulson2008} we only have to focus on rewriting the first line. First we write the term $\overline{S}(a,t)$ out in terms of the original quantities and we rewrite $\overline{Z}$ as a function of the age-specific average trait values:
\begin{equation}
\ages \frac{c(a,t)}{\overline{W}(t)} \avav{Z}{S}{a,t} - \overline{Z}(t) = \ages \frac{N(a,t)\overline{Z}(a,t)}{N(t+1)}\frac{\sum_{i,a_i=a} s_i}{N(a,t)} - \sum_{a=1}^{\omega} c(a,t) \overline{Z}(a,t)  
\end{equation}
Now we realise that $\sum_{i,a_i=a} s_i$ is equal to the number of individuals that survive the timestep and thus (since no individuals are newly born into higher age classes), we can write: $\sum_{i,a_i=a} s_i= N(a+1,t+1)$. However for $a=\omega$ this quantity is $0$ by definition. This now yields:
\begin{align}
\ages \frac{N(a,t)\overline{Z}(a,t)}{N(t+1)}\frac{\sum_{i,a_i=a} s_i}{N(a,t)} - \sum_{a=1}^{\omega} c(a,t) \overline{Z}(a,t) & = \sum_{a=1}^{\omega-1} c(a+1,t+1)\overline{Z}(a,t) - \sum_{a=1}^{\omega} c(a,t) \overline{Z}(a,t) \\ & = \sum_{a=1}^{\omega-1} \Delta c(a,t)\overline{Z}(a,t) - c(\omega,t) \overline{Z}(\omega,t)
\end{align}
with $\Delta c(a,t) = c(a+1,t+1) - c(a,t)$. Putting the equation together we find the age structured price equation:
\begin{align}
\Delta \overline{Z} & = \sum_{a=1}^{\omega-1} \Delta c(a,t)\overline{Z}(a,t) - c(\omega,t) \overline{Z}(\omega,t) \\ & + \ages \frac{c(a,t)}{\overline{W}(t)} \left( \cov{S}{Z}(a,t)  + \overline{SG}(a,t)\right) \notag \\ & + \ages \frac{c(a,t)}{\overline{W}(t)} \left( \avav{R}{D}{a,t} + \cov{R}{D}(a,t) + \avav{Z}{R}{a,t} + \cov{Z}{R}(a,t) \right) \notag
\label{appape4}
\end{align}

\subsection{Demographic contribution and choice of timesteps}\label{app:ape:step}
We shall illustrate the influence of the choice of timesteps with some simple made up data. In this data we shall only consider survival ($s_i$) and trait value (say body mass, $z_i$), assuming that individuals cannot grow, nor can they reproduce. This reduces the age structured price equation to:
\begin{equation}
\Delta \overline{Z} = \sum_{a=1}^{\omega-1} \Delta c(a,t)\overline{Z}(a,t) - c(\omega,t) \overline{Z}(\omega,t) + \ages \frac{c(a,t)}{\overline{W}(t)} \cov{S}{Z}(a,t)
\end{equation}
We consider a population of 10 individuals, each born in a different month. Measurement of the population takes place on the first day of each month. The column birth indicates in which month the individual was first seen, the column death indicates in which month the individual was found dead. \\
\begin{tabular}{rrrr}
  \hline
ID & birth & death & mass \\ 
  \hline
1 &   1 &  11 & 7.70 \\ 
  2 &   2 &  11 & 5.10 \\ 
  3 &   3 &  20 & 6.00 \\ 
  4 &   4 &  16 & 5.30 \\ 
  5 &   6 &  19 & 6.60 \\ 
  6 &   8 &  22 & 7.70 \\ 
  7 &   9 &  22 & 6.20 \\ 
  8 &  10 &  28 & 6.10 \\ 
  9 &  11 &  23 & 7.90 \\ 
  10 &  12 &  25 & 7.50 \\ 
   \hline
\end{tabular} \\
The question we ask is how the average body mass changes from month $13$ (that is january of the second year) to month $25$ (january of the next year). In month $13$ all individuals except number $1$ and $2$ are still alive. In month $25$ only individual $8$ is still alive. Over this timespan we note that:
\begin{equation}
\Delta \overline{Z} = \overline{Z}(25) - \overline{Z}(13) = 6.1 - 6.6625 = - 0.5625
\end{equation}
\subsubsection{Using a 1 year timestep}
On a one year basis, all individuals are in the same age group. On this basis we call month $13$ the start of the year $t=1$ and month $25$ the start of the year $t=25$. Hence, we see that ($s$ indicating survival to the next year):\\
\begin{tabular}{rrrr}
  \hline
 & mass & age & s \\ 
  \hline
3 & 6.00 &   1 &   0 \\ 
  4 & 5.30 &   1 &   0 \\ 
  5 & 6.60 &   1 &   0 \\ 
  6 & 7.70 &   1 &   0 \\ 
  7 & 6.20 &   1 &   0 \\ 
  8 & 6.10 &   1 &   1 \\ 
  9 & 7.90 &   1 &   0 \\ 
  10 & 7.50 &   1 &   0 \\ 
   \hline
\end{tabular} \\
We set $\omega=2$ (in years) since it is the maximum age that individuals can reach over this time step. Now the age structured price equation further reduces to:
\begin{equation}
\Delta \overline{Z} = (c(2,2)-c(1,1))\overline{Z}(1,1) - c(2,1) \overline{Z}(2,1) + \frac{c(1,1)}{\overline{W}(1)} \cov{S}{Z}(1,1) + \frac{c(2,1)}{\overline{W}(1)} \cov{S}{Z}(2,1)
\end{equation}
Now we note that:
\begin{align*}
c(1,1) = c(2,2) & = 1\\
c(2,1) & = 0 \\
\overline{W}(1) & = \frac{1}{8} \\
cov(S,Z)(1,1) = \overline{SZ}(1,1) - \overline{Z}(1,1)\overline{S}(1,1) & = -0.070\dots
\end{align*}
Plugging these numbers into the age structured price equation it is easy to see that the only contribution to $\Delta \overline{Z}$ comes from the survival selection term.
\begin{equation}
\Delta \overline{Z} = \frac{c(1,1)}{\overline{W}(1)}cov(S,Z)(1,1) = 8 \cdot -0.070\dots = -0.05625
\end{equation}
We can thus conclude that the change in trait value between month $13$ and month $25$ is completely caused by survival selection.
\subsubsection{Using a 1 month timestep}
When using a timestep with a length of 1 month, every single individual will have a unique age. Because of this, $cov(Z,S)(a,t)$ has to be $0$ for every single age class. All the changes in $\Delta Z$ just have to arise because of the other terms in the equation and we can thus - also - conclude that the change in body mass between month $13$ and month $25$ is entirely due to demography.\todo{If you wish I could perform the full calculation but I think it won't add much}
\subsection{Different terms}



\begin{equation}
\begin{aligned}
\Delta z (t) = & \underbrace{\sum_{a=1}^{\omega-1} \Delta c(a,t) \overline{z}(a,t) - c(\omega,t) \overline{z} (\omega,t) + \sum_{a=1}^\omega \frac{c(a,t}{\overline{W}(t)} \overline{r}(a,t)\overline{z}(a,t)}_{\text{demography}} \\ &+ \underbrace{\sum_{a=1}^{\omega} \frac{c(a,t)}{\overline{W}(t)} \left(cov(z,s)(a,t)+cov(z,r)(a,t)\right)}_{\text{selection}}+ \sum_{a=1}^\omega \frac{c(a,t}{\overline{W}(t)} \left(\overline{r}(a,t)\overline{d}(a,t) + \mathrm{cov}(d,r)(a,t) +\underbrace{\overline{sg}(a,t)}_{\text{plasticity}}\right) 
\end{aligned}
\label{keq1}
\end{equation}
Here:

\begin{tabular}{l p{0.7\textwidth}}
$c(a,t)$ & is the proportion of individuals of age $a$ in the population at time $t$ \\
$\Delta c(a,t)$ & is the difference between the proportion of individuals of age $a+1$ in the population at time $t+1$ and the proportion of individuals of age $a$. It thus measures how much the importance of a cohort in terms of population numbers changes from one timestep to the next.  \\
$\overline{z}(a,t)$ & is the average trait value for all individuals of age $a$ at time $t$.\\
$\omega$ & The maximum age\\
$\overline{W}(t)$ & The population growth rate from $t$ to $t+1$.\\
$\overline{r}(a,t)$ & The average number of offspring that an individual of age $a$ at time $t$ contributes to the population at time $t+1$.\\
$cov(z,s)(a,t)$ & The covariance between trait values and survival for all individuals of age $a$ at time $t$.\\
$cov(z,r)(a,t)$ & The covariance between the number of offspring and individual of age $a$ at time $t$ contributes to the population at time $t+1$ and the trait values of these parents.\\
$\overline{d}(a,t)$ & The average difference between the trait value of an individual with age $a$ at time $t$ and the average trait value of its offspring  \\
cov(d,r)(a,t) & The covariance between the number of offspring that an individual of age $a$ at time $t$ contributes to the population at time $t+1$ and how much the average trait value of the offspring from this individual \\
$\overline{sg}(a,t)$ & Average change in trait value for the individuals with age $a$ at time $t$, regardless of whether the individual survives. (For non-surviving individuals growth is set to 0).\\
\end{tabular}
