\section{Predictive ability}
\label{app:pred}

\subsection{IPM}
The IPM uses projections to estimate the contributions of the APE, either over the observed interval or further into the future, it can thus be used to do predictions, this in contrast to the APE. The quality of these predictions relies, however, on the the constancy of the statistical models that constitute the IPM.

By including the observed population structure, as we did in our application of the IPM, we focused on transient dynamics. IPMs have also been used to decompose trait dynamics by analysing asymptotic behaviour, by means of retrospective \citep{Ozgul2010} or prospective \citep{traill2014demography} perturbation analysis. However, if the considered population is still in a transient phase, asymptotic estimates (long-term predictions) may not be indicative for what is actually driving the trait dynamics. In these cases a transient analysis \citep[as outlined by][]{caswell2007} may be more appropriate. Another important difference between our application of the IPM framework and perturbation analysis is that, when performing a perturbation analysis, the effect of small changes in trait demography relationships is quantified. However, non-changing trait-demography relations can also affect trait dynamics, e.g. even non-changing survival selection will lead to changes in average body size. These points affect both predictions as well as the outcome of retrospective analyses.

\subsection{AM}
Also the AM can be used in a prospective manner, using either the breeder's equation or the Price equation \citep{Clayton1957,Lynch1998,Roff2007,Lynch2014}. As with the IPM, the accuracy of the predictions relies on the constancy of the occurring processes. In particular, in case of the AM, the stability of the genetic variance-covariance matrix of characters ($\boldsymbol{G}$-matrix) determines the accuracy in evolutionary predictions. Empirical studies investigating the stability of $\boldsymbol{G}$-matrices in the wild are still scarce, but it has been shown that they can be remarkably conserved among populations \citep{Roff2000} and among species \citep{Arnold2008}, as well as very stable over several decades within a population (\citealp{Garant2008}, but see \citealp{Bjorklund2013}). Nevertheless, predicting evolutionary trajectories in wild populations using the AM remains challenging \citep{Merila2001, Hadfield2010b, Morrissey2010} and there are too few studies to tell how useful the AM is for long-term projections in wild populations \citep{Teplitski2014}.