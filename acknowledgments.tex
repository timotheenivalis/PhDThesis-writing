\chapter{Acknowledgments}

Looking back to one's life, it is amazing to noticed how much the path it took has been influenced by a myriad of people. These acknowledgments are about a PhD thesis, but it is not clear how far they should go backward in time.

Erik Postma

Glauco Camenisch

Lukas Keller

%rest of the committee
Arpat Ozgul

Barbara Tschirren

Jarrod Hadfield

Marc K\'{e}ry

% Zurich people and related
%office
Pirmin Nietlisbach
PhilippXXX Becher
Judith Bachmann

Vanja Michel, Rien van XXX, JohanXXX

Koen van Benthem
Marjolein BXXX
Cindy Canale


%Scientific early experiences
I never had the opportunity to thank all the great people I have met since the beginning of my scientific life (because I didn't write a report for the two first internship, and acknowledgments were prohibited for my master thesis). Still, I owe them a lot, and the TRACE PERSISTANTE DE NOS RENCONTRE shape my PhD. 
In 2011, while I was trying to leave engineering for fundamental research, without having much clue of what it was about, Jean-Fran\c{c}ois Martin gave me the opportunity to give it a try, by tutoring a gap year entirely dedicated to research. Pierre-Andr\'{e} Crochet provided determinant help to find my way in the cloud of biology and reach a first research experiment, in the form of an internship at University of Oslo CEES. There, Glenn-Peter S{\ae}tre was an amazing first supervisor, extremely patient with my childish English, my misadventures in the lab and my perfect ignorance in evolutionary genetics, and was very supportive to find housing and survive one semester in Norway without any income. Thanks to him and all his team, especially Tore Oldeide Elgvin and Anna Fijarczyk, I had a great time that convinced that fundamental research in evolutionary biology would be a good way to spend my time. This might be where I learned the most about what the life of an evolutionary biologist was about: from sequencing DNA, to searching and understanding scientific literature; from the quasi-universal fascination of scientists for coffee and Friday beer, to the poetic insight of Kimura's neutral theory, and many more things. Eventually, thanks to them for making me publish my first research papers, although I did not do a lot and was probably not really understanding the little I did. 
I then went to Chiz\'{e} CEBC where I worked with a crew of amazing people including Adrien Pinot, Vincent Bretagnolle, David Pinaud, Vincent Lecoustre, Edoardo Tedesco, thanks to them and all the other Chiz\'eens for all they taught me. A particular thank to Mathieu Authier for his attempt to convert me to Bayesianism, and to Laurent Crespin for forcing me to go deeper into Mark-Recapture modeling and maximum of likelihood. Last but not least, Bertrand Gauffre was crucial in my scientific development, thanks to him for our many conversations, for taking me serious, for introducing me to Richard Dawkins' writings and for pushing me toward Montpellier. 
The B2E master in Montpellier was a time of hard studying and great fun during which some teachers opened my mind to new scientific horizons, thanks in particular to Patrice David, Isabelle Olivieri, Olivier Gimenez and Michel Raymond for that.
Thanks to Rapha\"{e}l Leblois for introducing me to the coalescent theory and its powerful thought experiments, as well as to C++ low-level programming. Coding \texttt{ForwardBackward} and its improved sequels was a crazy adventure, full of EPREUVES (Oh gosh, these bloody probabilities of identity in small sexual populations!) but mainly wonders. Thanks to Fran\c{c}ois Rousset for LAISSER ENTREVOIR UN MONDE DE RIGOUREUSE COMPREHENSION DU VIVANT.
Thanks again to PAC for his patient help explaining me again and again evolutionary concepts and writing tricks.
Also, thanks to Nicolas Bierne for his PhD offer and our fascinating discussion on oceanic streams, gene flow and the elusive eel, I still think about this alternative scientific and life path with curiosity and envy. 

Thanks to all the Montpellierains, in particular \'{E}meric Figuet, L\'{e}o Grasset, Joane Elleouet, Paul Sanders, Pascal Milesi, Julie Landes\dots our scientific conversations account for half what I know in evolutionary biology.
%My muses
Here start acknowledgments that go either beyond the scientific life.
Muses who pushed my scientific interest and helped me working hard (often by actually preventing me from working)
are Ga\"elle Duranthon, Manon Ghislain, L\'{e}o Grasset, Joane Elleouet (yes you are both acknowledged twice), Chelsea J. Little \dots

Thanks to my two adoptive families who kept me cheerful with meals, games, music, remote places, short nights and craziness during the last three years. These are the Goret family with an otter, a weasel, a canary, a baboon, a snail, a goat, a bee, a goose, a panda, a wolf, a bear, an owl, a lion and a chameleon; and the Margicon (\copyright \textit{une marque d\'{e}pos\'{e}e par terre}) family: Nounou, Chonchon, Caca, Doudou, Mirabelle and their mates/cats/penguins.

My early interest for ecology and evolution was much reinforced by walks and rides across the forests, meadows and moorlandsXXX of Lacaune mountains, by the contemplation of glittering and diverse carabids, by the excitement of long days monitoring bird migration, by dazzlingXXX conversations on conservation and nature management and by the constant realization that we still know so little and that it takes only some sweat and patience to discover new things.
This would have not been sustainable without the naturalists who I taught me so much and who kept me amazed. Thanks in particular to Amaury Calvet and other members of the ``Ligue pour la Protection des Oiseaux du Tarn'' as well as to the members of BeaOsea: Adrien Chaigne, Camille Denozière, Aur\'{e}lien Salesse, Denis Guillaumin and Manon Ghislain.

Some high school teachers are particularly influential not so much by what the taught, but rather by how they taught to learn, to see things in another way or by opening unexpected opportunities. Many would deserve acknowledgment for doing that at diverse degrees, but here are a few whose classes remain very alive: Beno\^{i}t Leviandier, Mdm Rolland, XXX math terminale, Jacky Cariou, Jean Leblanc.

Being born in such a family made life rather easy. Thanks especially to my parents Francine and Claude, to my siblings Kiyomi-\'{E}lodie and Cyrille, to my grand-parents Juliette, Louis, Andr\'{e} and Simone and to my uncle Laurent. They kept supporting, sustaining and trusting me over long years of studies they could less and less understand.