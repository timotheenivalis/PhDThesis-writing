\setchapterpreamble[ur][.6\textwidth]{%
\dictum[J.R.R. Tolkien, \textit{The Fellowship of the Ring}]{%
I don't know half of you half as well as I should like; and I like less than half of you half as well as you deserve.}\vskip1em
\dictum[Roman Kacew, dit Romain Gary, La Promesse de l'aube]{
Je feins l'adulte, mais, secr\`etement, je guette toujours le scarab\'ee d'or, et j'attends qu'un oiseau se pose sur mon \'epaule, pour me parler d'une voix humaine et me r\'ev\'eler enfin le pourquoi du comment.}\vskip1em
}

\chapter{Acknowledgments}

Looking back to one's life, it is amazing to notice how much the path it took has been influenced by a myriad of people. These acknowledgments are about a PhD thesis, but they are bound to look back in time far beyond the PhD onset.

First and foremost, many thanks to Erik Postma for being such a great supervisor. Erik set the track for a fascinating and successful PhD project when he realized the potential of the snow vole monitoring and wrote a thoughtful NSF proposal full of surprisingly correct guesses about the biology of the population. Erik taught me statistical techniques, ideas from quantitative genetics and presentation techniques. More importantly, though, he trained me to consider logical arguments critically, read scientific publications in depth (in particular during epics journal clubs that ended up ``rejecting'' more than half of what we read based on technical or logical flaws) and thus learn from the mistakes of others, to learn a bit less from mine. For four years, Erik's door was literally \emph{always} open (I cannot recall a time when he did not have some time to discuss) to answer my ``little questions'' and review my work.
Finally, it impossible not to mention that Erik introduced me to running, pushed me to run more often, longer and faster, and thus kept me fit, healthy and entertained.

Not being one of his students, and given his very tight schedule, I am especially grateful to Lukas Keller for long conversations, old paper mining and spontaneous suggestions. His seemingly encyclopaedic knowledge, and his almost as rich library, not only solved a few crucial issues relevant to my PhD, but also kept acute my appetite for the wonders of population genetics, organismic biology and statistics. 

%rest of the committee
Thanks to Arpat Ozgul for his constant enthusiasm about the project and for supporting its continuation during the postdoc to come. Also, thanks for being a bridge to the ideas and slang of population ecology and demography, thus clarifying ``translation'' issues. 

Thanks to Barbara Tschirren for her precious suggestions when trying to understand the voles in a more mechanistic way, molecular and physiological. Thanks for letting me take your brand new respirometer in the humidity, low pressures and shakiness of the field. 

The most important analyzes of my PhD were run with the package \texttt{MCMCglmm}, and I was extraordinarily lucky to count his creator, Jarrod Hadfield, among my committee members. Besides technical support, Jarrod helped clarifying some confusing concepts and results from quantitative genetics. Also, thanks to him for giving me the opportunity to review for the journal Evolution. 

Thanks to Marc K\'{e}ry for his help and encouragements with complicated JAGS models, but also for some sound life-planing advice. 

Thanks to Peter Wandeler for initiating the snow vole monitoring in Churwalden, while I was still a high-school student, unaware of the scree little happy folks. 
% Zurich people and related
%office

I could never thank enough Glauco Camenisch for the help he provided in the lab, on the field, in front of the computer and in the kitchen. Glauco saved me many weeks of work and made constant efforts to improve the quality of data set. He would deserve to be awarded a good share of this PhD.
With Glauco, the other backbone of the research group is Ursina Tobler. 
Ursina always keeps the administrative duties minimal on our side, but makes sure everything works fast and smoothly in the background. Thanks to her for making the live of students so easy. 

%Co-authors and field helpers
During the last three years, the majority of my scientific discussions were with Koen van Benthem. I am grateful for his many ridiculous questions and discoveries, most of them turned out to be puzzling or insightful. A particular thank for pushing me towards \LaTeX and Zurich Tonhalle. 
I wish he will one day understand the pattern behind prime numbers, and bring more mathematical rigor into ecology and evolution. 
Thanks also to him for taking the lead (i.e. first first authorship or sixth last authorship) on our ``Price's equation review'' (from which Price's equation was eventually extirpated), a long and tedious, but incredibly educational, collaboration, for which I also have to thank Marjolein Bruijning and Eelke Jongejans.

Vicente Garc\'ia-Navas can do research faster than a snow vole can run into a trap, and he publishes it in good journals before the apple is all eaten. It was very stimulating and good for my CV to work with him. 

Andres Hagmayer was an exemplar Master student, very serious, autonomous and efficient. Thanks to him for almost two years of a fruitful collaboration that taught me a lot about teaching. 

Thanks to Cindy Canale for helping with respirometry, for kind encouragements and nice parties. 
Many thanks to Dominique Waldvogel and Martina Schenkel for precious help on the scree. 


the ``residents'' of the office
Pirmin Nietlisbach
Philipp Becker
Judith Bachmann

This office was such a great place to work, live and laugh with the essential contribution of its intermittent occupants,
Erica Ponzi, 
Vanja Michel, 
Rien van Wijk and 
Johann Hegelbach.

%Inspiring relationships
Christine Grossen and Daniel Croll are among the kindest humans I have ever met. In particular, thanks for welcoming me during my first days in Z\"urich and for lending me a piano for two years.

Thanks to the Kokonuts\textemdash Hanna Kokko, Isobel Booksmythe, Ana\"is Tilquin, Nina Gerber, Susanne Schindler and Xiang-Yi Li\textemdash for inspiring journal clubs, where you can sit on an orange primitive salamander (definitely not a pony).

Many thanks (sadly impersonal so that the acknowledgements do not become a thesis chapter on their own right) to the Z\"urich folks for four very fun and enriching years. In a jumble, thanks to 
Alexandra Jansen van Rensburg 
, Alexander Nater
, Irene Weinberger
, Simon Evans
, Jasmin Winkler
, Joel Pick
, M\'elissa Lemoine
, Dennis Hansen
, Stefanie Muff
, Rassim Khelifa
, Debbie Leigh
, Josh van Buskirk 
, Kasia Sluzek
, Fr\'ed\'eric Guillaume
, Béatrice Nussberger
, Benedikt Gehr	
, Nina Vasiljevic
, Hedwig Ens
, Wolf Blanckenhorn
, Tina Cornioley
, Sam Cruickshank
, Mollie Brooks
, Juan Pablo Busso
, Franziska L\"orcher
, Jobran Chebib
, \emph{Cini} Gabriella Gall
, Andreas Sutter
, Aur\'elie Garnier
, Nino Magg
, Inge Juszak
, Gianalberto Losapio
\dots




Ashley E. Latimer.

%Scientific early experiences
I never had the opportunity to thank all the great people I have met since the beginning of my scientific life (because I didn't write a report for the two first internship, and acknowledgments were \textit{prohibited} (sic) for my master thesis). Still, I owe them a lot, and many people were instrumental in setting the path toward a PhD. 
In 2011, while I was trying to \sout{escape from} leave engineering for fundamental research, without having much clue of what it was about, Jean-Fran\c{c}ois Martin gave me the opportunity to give it a try by tutoring a gap year entirely dedicated to research. Pierre-Andr\'{e} Crochet provided determinant help to find my way in the cloud of biology and reach a first research experiment, in the form of an internship at University of Oslo CEES. There, Glenn-Peter S{\ae}tre was an amazing first supervisor, extremely patient with my childish English, my misadventures in the lab and my perfect ignorance in evolutionary genetics, and was very supportive to find housing and survive one semester in Norway without any income. Thanks to him and all his team, especially Tore Oldeide Elgvin and Anna Fijarczyk, I had a great time that convinced me that fundamental research in evolutionary biology would be a good way to spend my time. This might be where I learned the most about what the life of an evolutionary biologist was about: from sequencing DNA, to searching and understanding scientific literature; from the quasi-universal fascination of scientists for coffee and Friday beer, to the poetic insight of Kimura's neutral theory, and many more things. Eventually, thanks to them for making me publish my first research papers, although I did not do a lot and was probably not really understanding the little I did. 
I then went to Chiz\'{e} CEBC where I worked with a crew of amazing people including Adrien Pinot, Vincent Bretagnolle, David Pinaud, Vincent Lecoustre, Edoardo Tedesco, thanks to them and all the other Chiz\'eens for all they taught me. A particular thank to Mathieu Authier for his (partially-successful) attempt to convert me to Bayesianism, and to Laurent Crespin for forcing me to go deeper into Mark-Recapture modeling and maximum of likelihood. Last but not least, Bertrand Gauffre was crucial in my scientific development, thanks to him for our many conversations, for taking me seriously, for introducing me to Richard Dawkins' writings and for pushing me to do complete my Masters in Montpellier. 
The B2E Master in Montpellier was a time of hard studying, and great fun, during which some teachers opened my mind to new scientific horizons, thanks in particular to Patrice David, Isabelle Olivieri, Olivier Gimenez and Michel Raymond for that.
Thanks to Rapha\"{e}l Leblois for introducing me to the coalescent theory and its powerful thought experiments, as well as to \texttt{C++} programming. Coding \texttt{ForwardBackward} and its improved sequels was a crazy adventure, full of traps and wonders. Thanks to Fran\c{c}ois Rousset for giving me a small sight at what it means to understand population genetics, statistics, logic and the world in general.
Thanks again to PAC for his patient help explaining me again and again evolutionary concepts and writing tricks.
Also, thanks to Nicolas Bierne for his PhD offer and our fascinating discussion on oceanic streams, gene flow and the elusive eel, I still think about this alternative scientific and life path with curiosity and envy. 

Thanks to all the Montpellierains, in particular \'{E}meric Figuet, L\'{e}o Grasset, Joane Elleouet, Paul Sanders, Pascal Milesi, Julie Landes\dots our conversations account for of what I know in evolutionary biology.
%My muses
Out of the academic world, I could always count on the support of Ga\"elle Jeanne Duranthon during the last 12 years. Thanks also to my two adoptive families who kept me cheerful with meals, games, music, remote places, short nights and craziness during the last decade. These are the Goret family with an otter, a weasel, a canary, a baboon, a snail, a goat, a bee, a goose, a panda, a wolf, a bear, an owl, a lion and a chameleon; and the Margicon (\copyright \textit{une marque d\'{e}pos\'{e}e par terre}) family: Nounou, Chonchon, Caca, Doudou, Mirabelle and their mates/cats/penguins.

My early interest for ecology and evolution was much reinforced by wandering across the forests, meadows and moorlands of Lacaune mountains, by the contemplation of glittering and diverse carabids, by the excitement of long days monitoring bird migration, by astonishing conversations on conservation and nature management and by the constant realization that while we still know so little, it takes only some sweat and patience to discover new things.
This early encounter with life would have not been sustainable without the naturalists who taught me so much and kept me amazed. Thanks in particular to Amaury Calvet and other members of the ``Ligue pour la Protection des Oiseaux du Tarn'' as well as to the members of BeaOsea: Adrien Chaigne, Camille Denozière, Aur\'{e}lien Salesse, Denis Guillaumin and Manon Ghislain.

Some high school teachers were particularly influential not so much by what the taught, but rather by how they taught to learn, to see things in another way or by opening unexpected opportunities. Many would deserve acknowledgments for doing that at diverse degrees, but here are a few whose classes remain very alive: Beno\^{i}t Leviandier, Mdm Rolland, Jacky Cariou, Jean Leblanc.

Being born in such a family made life rather easy. Thanks especially to my parents Francine and Claude, to my siblings Kiyomi-\'{E}lodie and Cyrille, to my grand-parents Juliette, Louis, Andr\'{e} and Simone and to my uncle Laurent. They kept supporting, sustaining and trusting me over long years of studies they could less and less understand.