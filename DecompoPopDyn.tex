\chapter[Evolution or plasticity?]{Disentangling evolutionary, plastic and demographic processes underlying trait dynamics: A review of four frameworks}

Koen J. van Benthem*, Marjolein Bruijning*, \textbf{Timoth\'ee Bonnet}*, Eelke Jongejans$^\dagger$,
Erik Postma$^\dagger$, Arpat Ozgul$^\dagger$
\begin{itemize}
\item[*] co-first authors
\item[$\dagger$] co-last authors
\end{itemize}

\section{abstract}
\noindent \begin{enumerate}
\item Biologists are increasingly interested in decomposing trait dynamics into underlying processes, such as evolution, plasticity and demography. Four important frameworks that allow for such a decomposition are the quantitative genetic animal model (AM), the `Geber' method (GM), the age-structured Price equation (APE), and the integral projection model (IPM). However, as these frameworks have largely been developed independently, they differ in the assumptions they make, the data they require, as well as their outcomes and interpretation. 
\item Here we evaluate the way each of these frameworks decompose trait dynamics into underlying processes. To do so, we apply them to simulated data for a hypothetical animal population. We simulated scenarios with and without selection on body size, and with high and low heritability. In all four scenarios, body size also contained a plastic component.
\item The APE and IPM provided similar results, as did the AM and GM, with important differences between the former and the latter. All frameworks detected positive contributions of selection in the high but not in the low selection scenario. However, the APE and IPM did not distinguish between the high and low heritability scenarios, while the AM and GM did. Both the AM and GM revealed a high contribution of plasticity. In all scenarios, the APE and IPM attributed most of the change in body size to ontogenetic growth and inheritance, which include the effects of plasticity, maternal effects and heritability. We show how these apparent discrepancies are mostly due to the aims and definitions of the different frameworks. For example, the APE and IPM capture selection, whereas the AM and GM focus on the response to selection. Furthermore, the frameworks differ in the processes that are ascribed to plasticity and to their method for taking into account demography.
\item We conclude that because of the inherent differences among frameworks, no single framework provides the `true' contributions of evolution, plasticity and demography. However, with a thorough understanding of any of these frameworks, they will provide valuable biological insight. This work thus supports both future analysis as well as the careful interpretation of existing work.

\end{enumerate}

\section{Introduction}

%%%%%%%%%%%%%%%%%%%%%%%%%%%%%%%%%%%%%%%%%%%%%%%%%%%%%%%%%%%%%%%%%%%%%%%%%%%%%%

Understanding trait and population dynamics and how the two are intertwined is crucial for predicting population resilience and viability \parencite[e.g.][]{merila2014}. Hence, which processes shape population-level trait dynamics (i.e., changes in trait distributions over time) is a fundamental question in ecology and evolution, and one which is gaining in urgency given mounting concern regarding the consequences of anthropogenic environmental change for natural populations \parencite[e.g.][]{parmesan2006}.

Phenotypic trait distributions may be altered across generations by genetic (i.e. evolutionary) processes, as well as by non-genetic processes, such as phenotypic plasticity. Since the realisation that evolutionary and ecological processes may act on the same time scale, distinguishing between the role of evolution and plasticity has been the subject of a substantial body of research \parencite{Hairston2005,Gienapp2008,Post2009}. To complicate matters further, changes in the demographic structure of a population may additionally shape trait distributions \parencite{Coulson2008}. Hence, understanding and predicting trait dynamics ideally requires simultaneously taking into account all three processes \parencite{Pelletier2007,Schoener2011}. 

To date, four major frameworks aiming at distinguishing between the role of evolution, phenotypic plasticity and demography have been developed: 1) The quantitative genetic framework, particularly the animal model \parencite[AM; e.g.][]{Henderson1950}, 2) the `Geber method' \parencite[GM;][]{Hairston2005}, 3) the age-structured Price equation \parencite[APE;][]{Coulson2008}, and 4) the application of the APE in conjunction with an integral projection model \parencite[IPM;][]{easterling2000,Ellner2006,Coulson2010}. As of yet, several studies have tried to explicitly estimate the relative importance of evolution, plasticity and/or demography using one of these approaches \parencite[e.g.][]{Reale2003, Ezard2009, Ozgul2009, Rebke2010, Becks2012, Morrissey2012b}. Nevertheless, fully disentangling and quantifying evolutionary, ecological and demographic processes and thereby predicting the consequential trait dynamics has proven to be problematic \parencite{Gienapp2008,Schoener2011,merila2014}. At least some of these difficulties can be attributed to the large amounts of (individual-based) long-term data required, which are often unavailable for natural populations \parencite{Clutton-brock2010}. However, even if sufficient data are available, synthesis of the results from the four frameworks is hampered by the fact that they have been developed largely independently of each other. As a consequence, they differ in their focus and aims, and as we show here, they define biological processes in non-equivalent ways.  

Here we provide a comparison of the differences, similarities and complementarity of each of these four decomposition frameworks by applying them to the same simulated datasets and comparing their outcomes. Thereby, we evaluate how they quantify the role of different ecological and evolutionary mechanisms in shaping trait dynamics under a range of biological scenarios. Together with a critical review of the theory underlying each of the frameworks, we provide comprehensive insight into their underlying assumptions, as well as the conceptual differences and similarities. This provides a much needed overview of the suitability of each framework with respect to both research questions and data availability.

%%%%%%%%%%%%%%%%%%%%%%%%%%%%%%%%%%%%%%%%%%%%%%%%%%%%%%%%%%%%%%%%%%%%%%%%%%%%%%


\section{Applying the four frameworks}

%%%%%%%%%%%%%%%%%%%%%%%%%%%%%%%%%%%%%%%%%%%%%%%%%%%%%%%%%%%%%%%%%%%%%%%%%%%%%%

\subsection{Data simulation}

Although it comes with the loss of some biological realism, using simulated rather than empirical data enables us to evaluate the frameworks under different scenarios and allows for replication. Furthermore, it ensures perfect knowledge of the processes that shape trait dynamics, acting as a reference against which to compare the results of each framework. Finally, simulated data do not suffer from the complications introduced by missing data. 

Data were simulated using a two-sex individual-based model of a closed population of a hypothetical animal species, implemented in R \parencite{R}. Here, we provide a brief overview, while a more complete description can be found in SI \ref{app:simul}. We also provide the R code on \newline \verb|https://github.com/koenvanbenthem/Disentangling_Dynamics_IBM|. We simulated a single trait, body size $z$. Size at birth is determined by an individual's genotype (10 loci, with 10 alleles each and mendelian inheritance, more details in SI \ref{app:simul:gen}), the body size of its mother, and a stochastic component (drawn from a Gaussian distribution; SI \ref{app:simul:birth}). Ontogenetic growth results in an increase of body size with age. Growth rate is proportional to body size and decreases with age, and is further influenced by per-capita food availability (SI \ref{app:simul:growth}). Males were randomly assigned to females, who have a $50\%$ chance of becoming reproductive after one year and whose reproductive probability increases with age. The litter size that a female produces depends on per-capita food availability, a stochastic component, and body size (SI \ref{app:simul:repr}). Survival probability first increases with age, but starts decreasing after year five, reflecting senescence, and is further influenced by per-capita food availability and body size. Maximum age is 30 years. Furthermore, a trade-off exists between female reproduction and survival, i.e. reproducing at time $t$ decreases survival probability to time $t+1$ (SI \ref{app:simul:surv}). 

Fifty time steps were simulated, of which the first ten were discarded from further analyses to allow the age structure to stabilize (Fig. \ref{simdata:6}). After ten years, total food availability started to decline. As food is divided over all individuals alive in a given year, with some individuals randomly obtaining more than others, the decline affects population and trait dynamics through individual survival, growth and (female) reproductive success (SI \ref{app:simul:food}). The remaining data spanned 40 years (i.e. approximately 13 generations), which is comparable to the length of some of the field studies these frameworks have been applied to \parencite{Clutton-brock2010}. 

To evaluate the behaviour of the frameworks under different circumstances, we simulated four different scenarios. First, survival and fertility selection on body size was either present ($s_+$) or absent ($s_0$). Under the $s_+$ scenarios, there was a positive effect of body mass on survival and on litter size for mothers. Second, the relative importance of genetic variation in shaping body size, commonly measured as heritability, was either high ($h_+$) or low ($h_-$). This was done by using either of two pre-defined genotype-phenotype maps: one with big and one with small variation in the effects of alleles. Furthermore, the plastic component in birth size was varied antagonistically. The parameter values for each of the four scenarios (\sh, \sH, \Sh and \SH) can be found in SI \ref{app:scenarios}. To evaluate the effect of stochasticity, each scenario was replicated 100 times.

Figure \ref{simdata} provides an illustration of some of the key characteristics of the datasets simulated under each scenario. Despite a substantial amount of stochastic variation across replicates within each scenario, clear differences in trait and population dynamics are apparent. As expected, the $s_+$ scenarios show a positive relation between body size and annual fitness, calculated as the sum of the survival and litter size to $t+1$, whereas the $s_0$ scenarios do not (Fig. \ref{simdata:4}). Furthermore, the proportion of the phenotypic variance attributable to variance in the simulated genotypic values (i.e. broad-sense heritability $\text{H}^2$) was ca. 0.50 in the $h_+$ and 0.08 in the $h_-$ scenario.

Although in all scenarios population size first increased (until year 20) and then decreased (Fig. \ref{simdata:1}), the  population size averaged across replicates reached up to 322 and 334 individuals in scenarios \Sh and \SH, whereas in \sh and \sH the maximum average population size was 245 and 252 individuals, respectively. Mean body size first increased rapidly, but started to decrease in all scenarios after the tenth year (Fig. \ref{simdata:2}): in \sh with \Range{-0.47}{-1.45}{0.63 \; 95\% \; \text{range among replicates}}, in \sH with \Range{-0.46}{-1.59}{0.0.68}, in \Sh with \Range{-0.75}{-1.87}{0.08}, and in \SH with \Range{-0.16}{-1.12}{0.83}.

Genotypic values for birth size, however, continued to increase only in scenario \SH, whereby the change was \Range{0.62}{0.23}{1.04} (Fig. \ref{simdata:3}). In \Sh a smaller increase was observed \Range{0.08}{-0.074}{0.24}, whereas \sh and \sH show on average no change in genotypic values. Correspondingly, average birth size increased only in the \SH scenario, with \Range{0.58}{0.092}{1.11}, between year 11 and year 50 (Fig. \ref{simdata:5}). 

%%%%%%%%%%%%%%%%%%%%%%%%%%%%%%%%%%%%%%%%%%%%%%%%%%%%%%%%%%%%%%%%%%%%%%%%%%%%%%


\printbibliography[heading=subbibliography]
