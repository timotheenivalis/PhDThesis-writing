\setchapterpreamble[ur][.6\textwidth]{%
\dictum[Fyodor Dostoyevsky, \textit{The Idiot} (1868--9)]{%
One can't understand everything at once, we can't begin with perfection all at once! In order to reach perfection one must begin by being ignorant of a great deal. And if we understand things too quickly, perhaps we shan't understand them thoroughly.}\vskip1em

\dictum[dude]{%
things}\vskip1em}

\chapter[Chapter 1: General introduction]{General introduction}
\chaptermark{General introduction}
%%%%%%%%%%%%%%%%%%%%%%%%%%%%%%%%%%%%%%%%%%%%%%%%%%%%%%%%%%%%%%%%%%%%%%%%%%%%%%%%%%%%%%

%%%%%%%%%%%%%%%%%%%%%%%%%%%%%%%%%%%%%%%%%%%
\section{Variation in fitness}
%variation in evolution
Understanding variation among living things is the heart of evolutionary questioning. 

Darwin great discovery was to recognize the variation within species as the fuel generating the astonishing diversity of species themselves.

%fitness

%%%%%%%%%%%%%%%%%%%%%%%%%%%%%%%%%%%%%%%%%%%
\section{Quantitative genetics}

How to measure and make sense of genetic variation?
For over a century, there have been two main approaches. 
These can be summarized as ``bottom-up'' and ``top-down''.
The Mendelians and the biometricians

%%%%%%%%%%%%%%%%%%%%%%%%%%%%%%%%%%%%%%%%%%%
\section{This thesis}

\subsection{Objectives}

\subsection{Churwalden snow voles}

\subsection{Thesis outline}


