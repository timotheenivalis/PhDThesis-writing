\begin{summary}
\addcontentsline{toc}{chapter}{Summary}
\textbf{
This thesis investigates the stochastic and selective causes of variation in fitness components, and the evolutionary consequences of this variation in a wild rodent population. 
It shows the contemporary genetic evolution of body mass and decouples classic estimates of selection from adaptive evolution. 
}

The heart of evolutionary biology is understanding the variation in organisms. For over 150 years, researchers have documented the causes of within-species variation and how it contributes to speciation and explains the fit between organisms and their environment. 
Recently, increasing concerns regarding rapid anthropogenic changes have driven renewed investigation of how wild populations adapt to environmental change.
This new focus has revealed the difficulties measuring natural selection, disentangling evolution from plastic changes, and predicting evolutionary trajectories. 
For instance, there are few robust examples of contemporary evolution in wild populations, casting doubt on the possibility that evolution can rescue populations from rapid environmental change.
In this thesis, I investigate the causes of natural selection and evolution in a wild population of snow voles (\textit{Chionomys nivalis}). Thanks to 10 years of intensive individual-based monitoring and genotyping, knowledge of this population includes life-history, morphological data, and a high-resolution pedigree. This population is therefore among the best available worldwide to measure selection and evolution in action. 

The population is nevertheless relatively small and recent publications suggest that the evolutionary potential in small populations is effectively cancelled by stochasticity in fitness components. I assess the methods used in those publications and demonstrate that the variation in fitness components is not purely stochastic. Small populations, including these snow voles, show evolutionary potential. 
 
With collaborators, I then compare four common methodological frameworks to disentangle the contributions to phenotypic change of evolution, plasticity, and demography. We identify important discrepancies between the frameworks, partly originating from using different definitions, but also possessing intrinsically different capabilities. Among the considered frameworks only quantitative genetics can measure genetic change.

Applying methods from quantitative genetics to the snow vole population, I demonstrate that body mass evolved adaptively over the study period. I show that phenotypic estimates of selection are not predictive of genetic evolution: neither the mean selection nor its temporal variation are related to the rate of genetic evolution. This demonstrates that the dominant purely-phenotypic method used to measure selection risks measuring variation in nutritional status instead. Nevertheless, I employed quantitative genetics to identify the target of selection and obtain selection estimates in line with the observed genetic change 

This thesis establishes contemporary evolution in a wild population and shows that evolutionary responses to environmental change cannot be reliably estimated nor understood from purely-phenotypic methods; an explicit genetic approach is necessary. 

\end{summary}
