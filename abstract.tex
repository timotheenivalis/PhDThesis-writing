\begin{summary}
\textbf{
This thesis investigates the stochastic and selective causes of variation in fitness components, and the evolutionary consequences of this variation in a wild rodent population. 
It shows the contemporary genetic evolution of body mass and the uncoupling between classic estimates of selection and adaptive evolution. 
}

Understanding variation among living beings is the heart of evolutionary biology. For over 150 years, researchers have documented the causes of within-species variation and how this variation generates diversity among species as well as the fit between organisms and their environment. The question of how wild populations adapt to environmental changes is receiving renewed attention, because methodological advances allow deeper investigation, but also because of ever increasing concerns regarding rapid anthropogenic changes. This new research focus highlights the difficulties to measure natural selection, to disentangle evolution from plastic changes and to predict evolutionary trajectories. The scarcity of robust examples of contemporary evolution in the wild casts doubt on the possibility that evolution could play a role in the face of rapid environmental changes.
In this thesis, I investigate the causes of natural selection and evolution in wild population of snow voles (\textit{Chionomys nivalis}). Thanks to 10 years of individual-based monitoring and of genotyping, knowledge of this population includes life-history and morphological data, and a pedigree containing 90\% of the parent-offspring links. This population is therefore among the best ones worldwide to measure selection and evolution in action. 

The population is nevertheless relatively small and recent publications suggested that the evolutionary potential in such populations would be very limited by stochasticity in fitness components. I assess the methods used in these publications, and demonstrate that variation in fitness components is not purely stochastic. Small populations, including the snow voles, show evolutionary potential. 
 
With collaborators, I then compare four common methodological frameworks to disentangle the contributions of evolution, plasticity and demography to phenotypic changes. We identify important discrepancies between the frameworks, partly originating from the use of different definitions, but also showing different capabilities. Among the frameworks considered, quantitative genetics only can measure genetic change.

Applying methods from quantitative genetics to the snow vole population, I show adaptive evolution in body mass over the study period. I show that phenotypic estimates of selection are not predictive of this evolution: their mean value and their temporal variation are not related to the rate of evolution. This demonstrates that the dominant method to measure selection is at risk of measuring variation in nutritional status instead. Still, based on quantitative genetics, I identify the target of selection and obtain estimations in line with the observed genetic change. 

In conclusion, this thesis demonstrates contemporary evolution in a wild population and shows that evolutionary responses to environmental changes cannot be reliably estimated nor understood from purely phenotypic methods, an explicit genetic approach is necessary. 

\end{summary}
